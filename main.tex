\documentclass[xcolor={dvipsnames},aspectratio=169]{beamer}
\usepackage[utf8]{inputenc}

\usetheme{Madrid}
\usecolortheme{default}
\setbeamertemplate{enumerate items}[default]
\setbeamercolor*{structure}{bg=white,fg=black}

\definecolor{MQRed}{RGB}{166,25,46}       % Red
\definecolor{MQDeepRed}{RGB}{118,35,47}   % Deep Red
\definecolor{MQBrightRed}{RGB}{214,0,28}  % Bright Red
\definecolor{MQMagenta}{RGB}{198,0,126}   % Magenta
\definecolor{MQPurple}{RGB}{128,34,95}    % Purple
\definecolor{MQCharcoal}{RGB}{55,58,54}   % Charcoal
\definecolor{MQSand}{RGB}{214,210,196}    % Sand


\usepackage{times,url}

\setbeamertemplate{footline}{BERTopic summarisation of
student responses to an in-class survey --- CC-BY Ballsun-Stanton and Hurley}



%------------------------------------------------------------
%This block of code defines the information to appear in the
%Title page
\title[Berttopic] %optional
{BERTopic summarisation of
student responses to an in-class survey}

% \subtitle{A short story}

\author[Brian Ballsun-Stanton] % (optional)
{Brian Ballsun-Stanton (FoA) and Vincent Hurley (FoA, SSC)}


\date[13 December 2023] % (optional)
{Sociolinguistics through Corpus Research, December 2023\newline{}Presentation CC-BY}


%End of title page configuration block
%------------------------------------------------------------


\begin{document}

%The next statement creates the title page.
\frame{\titlepage}


%---------------------------------------------------------
%This block of code is for the table of contents after
%the title page
\begin{frame}
\frametitle{Table of Contents}
\tableofcontents
\end{frame}
%---------------------------------------------------------


\section{Generative AI in PICT2020 Assessments}

%---------------------------------------------------------
%Changing visivility of the text
\begin{frame}
\frametitle{Generative AI in FoA undergraduate teaching in 2023}

\begin{itemize}
    \item Faculty of Arts experiment in Generative AI 
    \begin{itemize}
        \item PICT2020 --- Policing and Crime
        \item GRMN1020 --- German 2
    \end{itemize}
    \item AI ``Enabled'' and AI ``Agnostic''
\end{itemize}

\end{frame}

\begin{frame}
\frametitle{PICT2020 --- Policing and Crime}

\begin{itemize}
    \item Authentic assessment: Goal of the unit is to deliver a 5 minute oral review brief
    \item Assessments:
    \begin{itemize}
        \item Media review (Interrogate with and without Bing Chat)
        \item Annotated Bibliography
        \item Peer review practice + AI feedback
        \item 5 minute oral review
        
    \end{itemize}
    \item AI ``Enabled'' and AI ``Agnostic'' assessment.
    \item Faculty's experiment in what happens when students are taught how to use and think about LLMs and then expected to apply that in every assessment.
\end{itemize}

\end{frame}


\begin{frame}
\frametitle{Problem: how did students feel about generative AI in their assessments?}

\begin{itemize}
    \item 
\end{itemize}

\end{frame}





%---------------------------------------------------------


% %---------------------------------------------------------
% %Example of the \pause command
% \begin{frame}
% In this slide \pause

% the text will be partially visible \pause

% And finally everything will be there
% \end{frame}
% %---------------------------------------------------------

% \section{Second section}

% %---------------------------------------------------------
% %Highlighting text
% \begin{frame}
% \frametitle{Sample frame title}

% In this slide, some important text will be
% \alert{highlighted} because it's important.
% Please, don't abuse it.

% \begin{block}{Remark}
% Sample text
% \end{block}

% \begin{alertblock}{Important theorem}
% Sample text in red box
% \end{alertblock}

% \begin{examples}
% Sample text in green box. The title of the block is ``Examples".
% \end{examples}
% \end{frame}
% %---------------------------------------------------------


% %---------------------------------------------------------
% %Two columns
% \begin{frame}
% \frametitle{Two-column slide}

% \begin{columns}

% \column{0.5\textwidth}
% This is a text in first column.
% $$E=mc^2$$
% \begin{itemize}
% \item First item
% \item Second item
% \end{itemize}

% \column{0.5\textwidth}
% This text will be in the second column
% and on a second tought this is a nice looking
% layout in some cases.
% \end{columns}
% \end{frame}
% %---------------------------------------------------------


\end{document}